\documentclass[12pt]{article}

% input
\bibliographystyle{ieeetr}
\usepackage[utf8]{inputenc}
%\usepackage{times} % font
\usepackage{lmodern} % scalable font
\usepackage{graphicx} % include graphics
\usepackage{amsmath} % align, nobreakdash
\usepackage[pdf,tmpdir]{graphviz} % digraph
\usepackage{fullpage} % book margins -> std margins
\usepackage{wrapfig} % wrapfigure
%\usepackage{moreverb} % verbatimtabinput
\usepackage{subcaption} % subcaptionbox
\usepackage[colorlinks]{hyperref} % pdf links
\usepackage{url} % url support
%\usepackage{comment} % comment
\usepackage{xfrac}

% code doesn't wrap
\usepackage[table]{xcolor}
\definecolor{light-gray}{gray}{0.95}
\newcommand{\code}[1]{\colorbox{light-gray}{\texttt{#1}}}

% create new commands
%\def\^#1{\textsuperscript{#1}}
%\def\!{\overline}
%\def\degree{\ensuremath{^\circ}}
\def\Scale{0.5}

% colourize titles
\definecolor{ilrblue}{RGB}{79,166,220}
\usepackage{titling}
\pretitle{\vspace{-3em}\fontfamily{\sfdefault}\fontsize{18bp}{18bp}\color{ilrblue}\selectfont}
\posttitle{\par\vspace{18bp}}
\preauthor{\normalfont\bfseries\selectfont\MakeUppercase}
\postauthor{\par\vspace{4bp}}
\predate{\normalfont\selectfont}
\postdate{\par\vspace{-8bp}}
\usepackage{titlesec}
\titleformat{\section}{\fontfamily{\sfdefault}\selectfont\normalsize\bfseries\color{ilrblue}\MakeUppercase}{\thesection}{1em}{}
\titleformat{\subsection}{\fontfamily{\sfdefault}\normalsize\bfseries\color{ilrblue}}{\thesubsection}{1em}{}
\titleformat{\subsubsection}{\fontfamily{\sfdefault}\normalsize\bfseries\color{ilrblue}\it}{\thesubsubsection}{1em}{}
% Ewww
\makeatletter
\renewenvironment{abstract}{%
    \if@twocolumn
      \section*{\abstractname}%
    \else \small %
      \begin{center}%
        {\bfseries\color{ilrblue} \abstractname\vspace{\z@}\vspace{-6bp}}%
      \end{center}%
      \quotation
    \fi}
    {\if@twocolumn\else\endquotation\fi}
\makeatother

% for hyperref
\hypersetup{
  linkcolor=ilrblue, % internal (figure) links
  urlcolor=ilrblue,
  filecolor=ilrblue,
  citecolor=ilrblue, % bibliography links
  pdfauthor={\@author},
  pdftitle={\@title},
  pdfsubject={\@title},
  pdfpagemode=UseNone
}

\author{Neil A. Edelman}
\title{Engineering an allocation-conscious hash-table}
\date{2022-02-22}

\begin{document}

\maketitle

\abstract{A bucket scheme where overlapping each entry with an index to the next entry gives us an inline-chained hash-table. The front entry in the bucket is closed, and all others are open, taken from unoccupied slots. This design can often be less expensive, performance-wise, but also to understand and maintain. Having a chained hash-table with the simplicity of allocation of an open-addressing hash-table is especially attractive for low-level languages, devices, and data.}

\section{Introduction}

Performance is a critical issue, but we are also concerned with usability. Specifically, without higher-level language support for native hash-tables or automatic garbage-collection, how hard is it to understand and maintain a general hash-table.

Overlapping each entry with an index to the next entry creates a hash-table that is self-contained in memory, yet behaves as a chained hash in the regime where the load factor is less-then one.\cite{knuth1998sorting} We call this inline-chaining to differentiate it from separate-chaining, where the buckets are objects with links between them.

This data structure is similar to that of coalesced-hashing\cite{williams1959handling}, but coalescing cannot occur. The closed entries that form the heads of buckets and the open entries that form a stack are orthogonal.

\section{Stars Example}

%\begin{wrapfigure}{r}{0.5\textwidth} %[!ht]
\begin{figure}
	\centering
	\subcaptionbox{Separate-chaining.\label{types:separate}}{%
		\digraph[scale=\Scale]{separate}{
	rankdir=LR;
	graph [truecolor=true, bgcolor=transparent];
	fontface=modern;
	node [shape=box, style=filled, fillcolor="Gray95"];
	hash [label=<<TABLE BORDER="0">
	<TR>
		<TD BORDER="0"><FONT FACE="Times-Italic">i</FONT></TD>
		<TD BORDER="0"><FONT FACE="Times-Bold">next</FONT></TD>
	</TR>
	<TR>
		<TD ALIGN="RIGHT" BGCOLOR="Gray90">0x0</TD>
	</TR>
	<TR>
		<TD ALIGN="RIGHT">0x1</TD>
		<TD PORT="1">\detokenize{⬤}</TD>
	</TR>
	<TR>
		<TD ALIGN="RIGHT" BGCOLOR="Gray90">0x2</TD>
	</TR>
	<TR>
		<TD ALIGN="RIGHT">0x3</TD>
		<TD PORT="3">\detokenize{⬤}</TD>
	</TR>
	<TR>
		<TD ALIGN="RIGHT" BGCOLOR="Gray90">0x4</TD>
		<TD PORT="4" BGCOLOR="Gray90">\detokenize{⬤}</TD>
	</TR>
	<TR>
		<TD ALIGN="RIGHT">0x5</TD>
		<TD PORT="5">\detokenize{⬤}</TD>
	</TR>
	<TR>
		<TD ALIGN="RIGHT" BGCOLOR="Gray90">0x6</TD>
	</TR>
	<TR>
		<TD ALIGN="RIGHT">0x7</TD>
	</TR>
	<TR>
		<TD ALIGN="RIGHT" BGCOLOR="Gray90">0x8</TD>
		<TD PORT="8" BGCOLOR="Gray90">\detokenize{⬤}</TD>
	</TR>
	<TR>
		<TD ALIGN="RIGHT">0x9</TD>
	</TR>
	<TR>
		<TD ALIGN="RIGHT" BGCOLOR="Gray90">0xa</TD>
		<TD PORT="10" BGCOLOR="Gray90">\detokenize{⬤}</TD>
	</TR>
	<TR>
		<TD ALIGN="RIGHT" BORDER="0">0xb</TD>
	</TR>
	<TR>
		<TD ALIGN="RIGHT" BGCOLOR="Gray90">0xc</TD>
		<TD PORT="12" BGCOLOR="Gray90">\detokenize{⬤}</TD>
	</TR>
	<TR>
		<TD ALIGN="RIGHT">0xd</TD>
	</TR>
	<TR>
		<TD ALIGN="RIGHT" BGCOLOR="Gray90">0xe</TD>
	</TR>
	<TR>
		<TD ALIGN="RIGHT">0xf</TD>
		<TD PORT="15">\detokenize{⬤}</TD>
	</TR>
</TABLE>>];
	e1 [label=<<TABLE BORDER="0">
	<TR>
		<TD ALIGN="RIGHT">0x91</TD>
		<TD ALIGN="LEFT">Castor</TD>
		<TD ALIGN="RIGHT">52</TD>
		<TD PORT="1">\detokenize{⬤}</TD>
	</TR>
</TABLE>>];
	e3 [label=<<TABLE BORDER="0">
	<TR>
		<TD ALIGN="RIGHT">0x3</TD>
		<TD ALIGN="LEFT">Deneb</TD>
		<TD ALIGN="RIGHT">2615</TD>
		<TD PORT="3">\detokenize{⬤}</TD>
	</TR>
</TABLE>>];
	e4 [label=<<TABLE BORDER="0">
	<TR>
		<TD ALIGN="RIGHT" BGCOLOR="Gray90">0x44</TD>
		<TD ALIGN="LEFT" BGCOLOR="Gray90">Sirius</TD>
		<TD ALIGN="RIGHT" BGCOLOR="Gray90">8.6</TD>
		<TD PORT="4" BGCOLOR="Gray90">\detokenize{⬤}</TD>
	</TR>
</TABLE>>];
	e5 [label=<<TABLE BORDER="0">
	<TR>
		<TD ALIGN="RIGHT">0x35</TD>
		<TD ALIGN="LEFT">Spica</TD>
		<TD ALIGN="RIGHT">250</TD>
		<TD PORT="5">\detokenize{⬤}</TD>
	</TR>
</TABLE>>];
	e8 [label=<<TABLE BORDER="0">
	<TR>
		<TD ALIGN="RIGHT" BGCOLOR="Gray90">0xd8</TD>
		<TD ALIGN="LEFT" BGCOLOR="Gray90">Rigel</TD>
		<TD ALIGN="RIGHT" BGCOLOR="Gray90">860</TD>
		<TD PORT="8" BGCOLOR="Gray90">\detokenize{⬤}</TD>
	</TR>
</TABLE>>];
	e10 [label=<<TABLE BORDER="0">
	<TR>
		<TD ALIGN="RIGHT" BGCOLOR="Gray90">0x4a</TD>
		<TD ALIGN="LEFT" BGCOLOR="Gray90">Betelgeuse</TD>
		<TD ALIGN="RIGHT" BGCOLOR="Gray90">700</TD>
		<TD PORT="10" BGCOLOR="Gray90">\detokenize{⬤}</TD>
	</TR>
</TABLE>>];
	e11 [label=<<TABLE BORDER="0">
	<TR>
		<TD ALIGN="RIGHT">0x4f</TD>
		<TD ALIGN="LEFT">Procyon</TD>
		<TD ALIGN="RIGHT">11</TD>
		<TD PORT="11">\detokenize{◯}</TD>
	</TR>
</TABLE>>];
	e12 [label=<<TABLE BORDER="0">
	<TR>
		<TD ALIGN="RIGHT" BGCOLOR="Gray90">0xec</TD>
		<TD ALIGN="LEFT" BGCOLOR="Gray90">Regulus</TD>
		<TD ALIGN="RIGHT" BGCOLOR="Gray90">79</TD>
		<TD PORT="12" BGCOLOR="Gray90">\detokenize{⬤}</TD>
	</TR>
</TABLE>>];
	e13 [label=<<TABLE BORDER="0">
	<TR>
		<TD ALIGN="RIGHT">0x33</TD>
		<TD ALIGN="LEFT">Antares</TD>
		<TD ALIGN="RIGHT">550</TD>
		<TD PORT="13">\detokenize{◯}</TD>
	</TR>
</TABLE>>];
	e14 [label=<<TABLE BORDER="0">
	<TR>
		<TD ALIGN="RIGHT">0xb3</TD>
		<TD ALIGN="LEFT">Sol</TD>
		<TD ALIGN="RIGHT">0</TD>
		<TD PORT="14">\detokenize{◯}</TD>
	</TR>
</TABLE>>];
	e15 [label=<<TABLE BORDER="0">
	<TR>
		<TD ALIGN="RIGHT">0x9f</TD>
		<TD ALIGN="LEFT">Polaris</TD>
		<TD ALIGN="RIGHT">430</TD>
		<TD PORT="15">\detokenize{⬤}</TD>
	</TR>
</TABLE>>];
	node [shape=plain, fillcolor=none, headclip = false, tailclip=false]
	hash:1 -> e1;
	hash:3 -> e3 -> e13 -> e14;
	hash:4 -> e4;
	hash:5 -> e5;
	hash:8 -> e8;
	hash:10 -> e10;
	hash:12 -> e12;
	hash:15 -> e15;
	e15 -> e11;
		}
	}
	\subcaptionbox{Open-addressing.\label{types:open}}{%
		\digraph[scale=\Scale]{open}{
	rankdir=LR;
	graph [truecolor=true, bgcolor=transparent];
	fontface=modern;
	node [shape=box, style=filled, fillcolor="Gray95"];
	hash [label=<<TABLE BORDER="0">
	<TR>
		<TD BORDER="0"><FONT FACE="Times-Italic">i</FONT></TD>
		<TD BORDER="0"><FONT FACE="Times-Italic">disp.</FONT></TD>
		<TD BORDER="0"><FONT FACE="Times-Bold">hash</FONT></TD>
		<TD BORDER="0"><FONT FACE="Times-Bold">key</FONT></TD>
		<TD BORDER="0"><FONT FACE="Times-Bold">value</FONT></TD>
	</TR>
	<TR>
		<TD ALIGN="RIGHT" BGCOLOR="Gray90">0x0</TD>
		<TD BGCOLOR="Gray90">1</TD>
		<TD ALIGN="RIGHT" BGCOLOR="Gray90">0x4f</TD>
		<TD ALIGN="LEFT" BGCOLOR="Gray90">Procyon</TD>
		<TD ALIGN="RIGHT" BGCOLOR="Gray90">11</TD>
	</TR>
	<TR>
		<TD ALIGN="RIGHT">0x1</TD>
		<TD>0</TD>
		<TD ALIGN="RIGHT">0x91</TD>
		<TD ALIGN="LEFT">Castor</TD>
		<TD ALIGN="RIGHT">52</TD>
	</TR>
	<TR>
		<TD ALIGN="RIGHT" BGCOLOR="Gray90">0x2</TD>
	</TR>
	<TR>
		<TD ALIGN="RIGHT">0x3</TD>
		<TD>0</TD>
		<TD ALIGN="RIGHT">0x3</TD>
		<TD ALIGN="LEFT">Deneb</TD>
		<TD ALIGN="RIGHT">2615</TD>
	</TR>
	<TR>
		<TD ALIGN="RIGHT" BGCOLOR="Gray90">0x4</TD>
		<TD BGCOLOR="Gray90">1</TD>
		<TD ALIGN="RIGHT" BGCOLOR="Gray90">0x33</TD>
		<TD ALIGN="LEFT" BGCOLOR="Gray90">Antares</TD>
		<TD ALIGN="RIGHT" BGCOLOR="Gray90">550</TD>
	</TR>
	<TR>
		<TD ALIGN="RIGHT">0x5</TD>
		<TD PORT="14">2</TD>
		<TD ALIGN="RIGHT">0xb3</TD>
		<TD ALIGN="LEFT">Sol</TD>
		<TD ALIGN="RIGHT">0</TD>
	</TR>
	<TR>
		<TD ALIGN="RIGHT" BGCOLOR="Gray90">0x6</TD>
		<TD BGCOLOR="Gray90">2</TD>
		<TD ALIGN="RIGHT" BGCOLOR="Gray90">0x44</TD>
		<TD ALIGN="LEFT" BGCOLOR="Gray90">Sirius</TD>
		<TD ALIGN="RIGHT" BGCOLOR="Gray90">8.6</TD>
	</TR>
	<TR>
		<TD ALIGN="RIGHT">0x7</TD>
		<TD>2</TD>
		<TD ALIGN="RIGHT">0x35</TD>
		<TD ALIGN="LEFT">Spica</TD>
		<TD ALIGN="RIGHT">250</TD>
	</TR>
	<TR>
		<TD ALIGN="RIGHT" BGCOLOR="Gray90">0x8</TD>
		<TD PORT="8" BGCOLOR="Gray90">0</TD>
		<TD ALIGN="RIGHT" BGCOLOR="Gray90">0xd8</TD>
		<TD ALIGN="LEFT" BGCOLOR="Gray90">Rigel</TD>
		<TD ALIGN="RIGHT" BGCOLOR="Gray90">860</TD>
	</TR>
	<TR>
		<TD ALIGN="RIGHT">0x9</TD>
	</TR>
	<TR>
		<TD ALIGN="RIGHT" BGCOLOR="Gray90">0xa</TD>
		<TD PORT="10" BGCOLOR="Gray90">0</TD>
		<TD ALIGN="RIGHT" BGCOLOR="Gray90">0x4a</TD>
		<TD ALIGN="LEFT" BGCOLOR="Gray90">Betelgeuse</TD>
		<TD ALIGN="RIGHT" BGCOLOR="Gray90">700</TD>
	</TR>
	<TR>
		<TD ALIGN="RIGHT">0xb</TD>
	</TR>
	<TR>
		<TD ALIGN="RIGHT" BGCOLOR="Gray90">0xc</TD>
		<TD PORT="12" BGCOLOR="Gray90">0</TD>
		<TD ALIGN="RIGHT" BGCOLOR="Gray90">0xec</TD>
		<TD ALIGN="LEFT" BGCOLOR="Gray90">Regulus</TD>
		<TD ALIGN="RIGHT" BGCOLOR="Gray90">79</TD>
	</TR>
	<TR>
		<TD ALIGN="RIGHT">0xd</TD>
	</TR>
	<TR>
		<TD ALIGN="RIGHT" BGCOLOR="Gray90">0xe</TD>
	</TR>
	<TR>
		<TD ALIGN="RIGHT">0xf</TD>
		<TD PORT="15">0</TD>
		<TD ALIGN="RIGHT">0x9f</TD>
		<TD ALIGN="LEFT">Polaris</TD>
		<TD ALIGN="RIGHT">430</TD>
	</TR>
</TABLE>>];
	node [shape=plain, fillcolor=none]
		}
	}
	\subcaptionbox{Inline-chaining.\label{types:inline}}{%
		\digraph[scale=\Scale]{inline}{
	rankdir=LR;
	graph [truecolor=true, bgcolor=transparent];
	fontface=modern;
	node [shape=box, style=filled, fillcolor="Gray95"];
	hash [label=<<TABLE BORDER="0">
	<TR>
		<TD BORDER="0"><FONT FACE="Times-Italic">i</FONT></TD>
		<TD BORDER="0"><FONT FACE="Times-Bold">hash</FONT></TD>
		<TD BORDER="0"><FONT FACE="Times-Bold">key</FONT></TD>
		<TD BORDER="0"><FONT FACE="Times-Bold">value</FONT></TD>
		<TD BORDER="0"><FONT FACE="Times-Bold">next</FONT></TD>
	</TR>
	<TR>
		<TD ALIGN="RIGHT" BGCOLOR="Gray90">0x0</TD>
	</TR>
	<TR>
		<TD ALIGN="RIGHT">0x1</TD>
		<TD ALIGN="RIGHT">0x91</TD>
		<TD ALIGN="LEFT">Castor</TD>
		<TD ALIGN="RIGHT">52</TD>
		<TD PORT="1">\detokenize{⬤}</TD>
	</TR>
	<TR>
		<TD ALIGN="RIGHT" BGCOLOR="Gray90">0x2</TD>
	</TR>
	<TR>
		<TD ALIGN="RIGHT">0x3</TD>
		<TD ALIGN="RIGHT">0x3</TD>
		<TD ALIGN="LEFT">Deneb</TD>
		<TD ALIGN="RIGHT">2615</TD>
		<TD PORT="3">\detokenize{⬤}</TD>
	</TR>
	<TR>
		<TD ALIGN="RIGHT" BGCOLOR="Gray90">0x4</TD>
		<TD ALIGN="RIGHT" BGCOLOR="Gray90">0x44</TD>
		<TD ALIGN="LEFT" BGCOLOR="Gray90">Sirius</TD>
		<TD ALIGN="RIGHT" BGCOLOR="Gray90">8.6</TD>
		<TD PORT="4" BGCOLOR="Gray90">\detokenize{⬤}</TD>
	</TR>
	<TR>
		<TD ALIGN="RIGHT">0x5</TD>
		<TD ALIGN="RIGHT">0x35</TD>
		<TD ALIGN="LEFT">Spica</TD>
		<TD ALIGN="RIGHT">250</TD>
		<TD PORT="5">\detokenize{⬤}</TD>
	</TR>
	<TR>
		<TD ALIGN="RIGHT" BGCOLOR="Gray90">0x6</TD>
	</TR>
	<TR>
		<TD ALIGN="RIGHT">0x7</TD>
	</TR>
	<TR>
		<TD ALIGN="RIGHT" BGCOLOR="Gray90">0x8</TD>
		<TD ALIGN="RIGHT" BGCOLOR="Gray90">0xd8</TD>
		<TD ALIGN="LEFT" BGCOLOR="Gray90">Rigel</TD>
		<TD ALIGN="RIGHT" BGCOLOR="Gray90">860</TD>
		<TD PORT="8" BGCOLOR="Gray90">\detokenize{⬤}</TD>
	</TR>
	<TR>
		<TD ALIGN="RIGHT">0x9</TD>
	</TR>
	<TR>
		<TD ALIGN="RIGHT" BGCOLOR="Gray90">0xa</TD>
		<TD ALIGN="RIGHT" BGCOLOR="Gray90">0x4a</TD>
		<TD ALIGN="LEFT" BGCOLOR="Gray90">Betelgeuse</TD>
		<TD ALIGN="RIGHT" BGCOLOR="Gray90">700</TD>
		<TD PORT="10" BGCOLOR="Gray90">\detokenize{⬤}</TD>
	</TR>
	<TR>
		<TD ALIGN="RIGHT" BORDER="2">0xb</TD>
		<TD ALIGN="RIGHT">0x4f</TD>
		<TD ALIGN="LEFT">Procyon</TD>
		<TD ALIGN="RIGHT">11</TD>
		<TD PORT="11">\detokenize{◯}</TD>
	</TR>
	<TR>
		<TD ALIGN="RIGHT" BGCOLOR="Gray90">0xc</TD>
		<TD ALIGN="RIGHT" BGCOLOR="Gray90">0xec</TD>
		<TD ALIGN="LEFT" BGCOLOR="Gray90">Regulus</TD>
		<TD ALIGN="RIGHT" BGCOLOR="Gray90">79</TD>
		<TD PORT="12" BGCOLOR="Gray90">\detokenize{⬤}</TD>
	</TR>
	<TR>
		<TD ALIGN="RIGHT">0xd</TD>
		<TD ALIGN="RIGHT">0x33</TD>
		<TD ALIGN="LEFT">Antares</TD>
		<TD ALIGN="RIGHT">550</TD>
		<TD PORT="13">\detokenize{◯}</TD>
	</TR>
	<TR>
		<TD ALIGN="RIGHT" BGCOLOR="Gray90">0xe</TD>
		<TD ALIGN="RIGHT" BGCOLOR="Gray90">0xb3</TD>
		<TD ALIGN="LEFT" BGCOLOR="Gray90">Sol</TD>
		<TD ALIGN="RIGHT" BGCOLOR="Gray90">0</TD>
		<TD PORT="14" BGCOLOR="Gray90">\detokenize{◯}</TD>
	</TR>
	<TR>
		<TD ALIGN="RIGHT">0xf</TD>
		<TD ALIGN="RIGHT">0x9f</TD>
		<TD ALIGN="LEFT">Polaris</TD>
		<TD ALIGN="RIGHT">430</TD>
		<TD PORT="15">\detokenize{⬤}</TD>
	</TR>
</TABLE>>];
	node [shape=plain, fillcolor=none]
	e13 [label="0xd"];
	hash:3 -> e13 [tailclip=false];
	e14 [label="0xe"];
	e13 -> e14;
	i0xe [label="0xe", fontcolor="Gray"];
	hash:13 -> i0xe [color="Gray"];
	e11 [label="0xb"];
	hash:15 -> e11 [tailclip=false];
		}
	}
	\caption{A map from some star names to light-year distance using different collision-resolution schemes. Order of insertion: Sol, Sirius, Rigel, Procyon, Betelgeuse, Antares, Spica, Deneb, Regulus, Castor, Polaris. Load factor 11/16 = 0.69. Expected value of number of queries for chaining, 1.36(0.7), and for open-addressing 1.60(0.9), (sample standard deviation.)\label{types}}
\end{figure}

Figure~\ref{types} shows a comparison of some standard hash-table types. It uses D.J.~Bernstein's {\it djb2} to hash a string to 8-bit unsigned integer. All tables use a most-recently-used heuristic as probe-order; experimentally, this was found to make little difference in the run-time, and is advantageous when the access pattern is non-uniform\cite{brent1973reducing, sleator1985self}.

Separate-chaining is seen in Figure~\ref{types:separate}; this is more like T.D. Hanson's {\it uthash}: only being in one hash-table at a time. A similar hash to {\it C++}'s {\it std::unordered\_map}, {\it Lua}'s {\it table}, and many others, would have another dereference between the linked-list and the entry. This style of hash-table allows unconstrained load factors. With ordered data, keeping a self-balancing tree cuts down the worst case to $\mathcal{O}(\log n)$\cite{knuth1998sorting}, as in {\it Java}. The expected number of dereferences is a constant added to the number of queries.

Open-addressing\cite{peterson1957addressing} as seen in Figure~\ref{types:open}, is another, more compact, and generally more cache-coherent table design. Robin Hood hashing\cite{celis1985robin} has been used to keep the variation in the query length to a minimum; here, with the condition on whether to evict strengthened because of the most-recently-used heuristic. It has lower maximum load-factor because clustering decreases performance as the load-factor reaches saturation\cite{skiena2008algorithm}. Practically, 0.69 is high; {\it Python}\cite{knuth1998sorting} uses a maximum of \sfrac{2}{3}. One can calculate the displacement from the hash, but we have to have a general way of telling if it's null. The lack of symmetry presents a difficulty removing entries.

Inline-chaining, as seen in Figure~\ref{types:inline}, is, in many ways, a hybid between the two. The expected number of dereferences is number of queries for simple data. Because of the {\it next index} field, the space taken is one index {\it per} entry more then open-addressing. This offers a convenient place to store a flag for null. Being chained offers a much higher load-factor, but it is not possible to exceed one. The dark circles represent the closed heads, and the outline the open stack, with a highlight to indicate the stack position.

\section{Implementation}

\subsection{Next Entry}

In order to offer a complete range of data for the key, (specifically, not just a pointer, it could be an integer zero,) it is convenient to store flags in the `next' field. There are two special values that must be differentiated: there is no closed value associated with this address, called `NULL', and there is no next value in the bucket, called `END'.

Since the implementation uses power-of-two resizes, it causes the addressable space to be one-bit less; half the size, or the equivalent of a signed positive integer. The bits `NULL' and `END' were chosen to minimize average power requirements while leaving a natural $[0, 011..11]$ for addressing. That is, $100..00$, and $100..01$. In Figure~\ref{types:inline}, where there are 8~bits, the addressable space is $[0, 127]$, and `NULL', `END' are 128, 129.

\subsection{Load Factor}

It is possible to get rid of load factor calculations by making the maximum load factor identically one. Only if the hash-table is full does one resize. In an inline-chained hash-table, the entries and the next index are overlapped; this means that the capacity for each goes up in the same allocation.

\subsection{Orthogonal}

We must move them around. Still O(n), but the average is twice that. Doesn't make much difference experimentally.

\subsection{Iteration}

For implementations that do not provide a special iteration mechanism, iteration on separate-chaining is $\mathcal{O}(\text{capacity} + \text{size})$. For open-addressing and inline-chaining, because we store collisions in the hash-table itself, it's $\mathcal{O}(\text{capacity})$. However, a practical capacity in open-addressing will be larger because the decrease in performance as the load-factor reaches saturation.

\subsection{Power-of-Two}

Consistent hashing.\cite{karger1997consistent}

/* Initialize new values. Mask to identify the added bits. */
	/* Rehash most closed buckets in the lower half. Create waiting
	 linked-stack by borrowing next. */
	/* Search waiting stack for buckets that moved concurrently. */
	/* Rebuild the top stack at the high numbers from the waiting at low. */

\subsection{Stack}

Lazy stack; hysteresis.
static void PN(forcestack)(struct N(table) *const table);
static void PN(shrinkstack)(struct N(table) *const table,
	const PN(uint) b);
static void PN(movetotop)(struct N(table) *const table, const PN(uint) m);

\subsection{Inverse Hash Function}

``Defining `TABLE\_INVERSE` says <typedef:<PN>hash\_fn> forms a bijection
 between the range in <typedef:<PN>key> and the image in <typedef:<PN>uint>.
 The keys are not stored in the hash table, but they are generated from the
 hashes using this inverse-mapping.'' otherwise ``Equivalence relation between <typedef:<PN>key> that satisfies
 `<PN>is\_equal\_fn(a, b) -> <PN>hash(a) == <PN>hash(b)`. Not used if
 `TABLE\_INVERSE` because the comparison is done in hash space, in that case.''
 
\subsection{Running Time}

The asymptotic worst-case run-time will be the same as any chained hash. However, the average 2.

\section{Conclusion}

\bibliography{table}

\end{document}
