\documentclass[12pt]{article}

% input
\bibliographystyle{ieeetr}
\usepackage[utf8]{inputenc}
%\usepackage{times} % font
\usepackage{lmodern} % scalable font
\usepackage{graphicx} % include graphics
\usepackage{amsmath} % align, nobreakdash
\usepackage[pdf,tmpdir]{graphviz} % digraph
\usepackage{fullpage} % book margins -> std margins
\usepackage{wrapfig} % wrapfigure
%\usepackage{moreverb} % verbatimtabinput
\usepackage{subcaption} % subcaptionbox
\usepackage[colorlinks]{hyperref} % pdf links
\usepackage{url} % url support
%\usepackage{comment} % comment

% code doesn't wrap
\usepackage[table]{xcolor}
\definecolor{light-gray}{gray}{0.95}
\newcommand{\code}[1]{\colorbox{light-gray}{\texttt{#1}}}

% create new commands
%\def\^#1{\textsuperscript{#1}}
%\def\!{\overline}
%\def\degree{\ensuremath{^\circ}}
\def\Scale{0.5}

% colourize titles
\definecolor{ilrblue}{RGB}{79,166,220}
\usepackage{titling}
\pretitle{\vspace{-3em}\fontfamily{\sfdefault}\fontsize{18bp}{18bp}\color{ilrblue}\selectfont}
\posttitle{\par\vspace{18bp}}
\preauthor{\normalfont\bfseries\selectfont\MakeUppercase}
\postauthor{\par\vspace{4bp}}
\predate{\normalfont\selectfont}
\postdate{\par\vspace{-8bp}}
\usepackage{titlesec}
\titleformat{\section}{\fontfamily{\sfdefault}\selectfont\normalsize\bfseries\color{ilrblue}\MakeUppercase}{\thesection}{1em}{}
\titleformat{\subsection}{\fontfamily{\sfdefault}\normalsize\bfseries\color{ilrblue}}{\thesubsection}{1em}{}
\titleformat{\subsubsection}{\fontfamily{\sfdefault}\normalsize\bfseries\color{ilrblue}\it}{\thesubsubsection}{1em}{}
% Ewww
\makeatletter
\renewenvironment{abstract}{%
    \if@twocolumn
      \section*{\abstractname}%
    \else \small %
      \begin{center}%
        {\bfseries\color{ilrblue} \abstractname\vspace{\z@}\vspace{-8bp}}%
      \end{center}%
      \quotation
    \fi}
    {\if@twocolumn\else\endquotation\fi}
\makeatother

% for hyperref
\hypersetup{
  linkcolor=ilrblue, % internal (figure) links
  urlcolor=ilrblue,
  filecolor=ilrblue,
  citecolor=ilrblue, % bibliography links
  pdfauthor={\@author},
  pdftitle={\@title},
  pdfsubject={\@title},
  pdfpagemode=UseNone
}

\author{Neil A. Edelman}
\title{Compact binary prefix trees}
\date{2021-10-20}

\begin{document}

\maketitle

\abstract{Our prefix-tree, digital-tree, or trie is an ordered set or map with key strings. We build a dynamic index of two-bytes {\it per} entry, only storing differences in a compact binary radix tree. To maximize locality of reference while descending the trie and minimizing update data, these are grouped together in a forest of fix-sized trees.}

\section{Introduction}

A trie is a tree that stores partitioned sets of strings\cite{de1959file, fredkin1960trie, jacquet1991analysis, askitis2011redesigning} so that, ``instead of basing a search method on comparisons between keys, we can make use of their representation as a sequence of digits or alphabetic characters [directly].\cite{knuth1997sorting}'' It is necessarily ordered, and allows prefix range queries.

Often, only parts of the key string are important; a radix trie (compact prefix tree) skips past the parts that are not important, as \cite{askitis2007hat}. If a candidate key match is found, a full match can be made with one index from the trie.

For most applications, a 256-ary trie is space-intensive; the index contains many spaces for keys that are unused. Various compression schemes are available, such as re-using a pool of memory\cite{de1959file}, reducing our encoding alphabet, or take smaller than 8-bit chunks\cite{fredkin1960trie}.

We use a combination binary radix trie, described in \cite{morrison1968patricia} as the PATRICIA automaton. Rather than being sparse, a Patricia-tree is a packed index. Recursively, it encodes which is next distinguishing bit and how many keys are on the zero path and the one path.

In practice, we talk about a string always terminated by a sentinel; this is an easy way to allow a string and it's prefix in the same trie\cite{morrison1968patricia}. In C, a NUL-terminated string automatically has this property, and is ordered correctly. Keys are sorted in lexicographic order by numerical value; \code{strcmp}-order, not by any collation algorithm.

\section{Implementation}

\subsection{Encoding}

% star-3
%\begin{wrapfigure}{r}{0.5\textwidth} %[!ht]
\begin{figure}
	\centering
	\subcaptionbox{Bit meaning.\label{star-0:bits}}{%
\digraph[scale=0.6]{star0bits}{
	graph [truecolor=true, bgcolor=transparent, fontname=modern];
	node [shape=none, fontname=modern];
	tree0x100a04120branch0 [label = <
<table border="0" cellspacing="0">
	<tr>
		<td align="left" port="0">Achernar<font color="Grey75">\detokenize{⊔}</font></td>
		<td>0</td>
		<td>1</td>
		<td>0</td>
		<td bgcolor="Grey95" border="1">0</td>
		<td>0</td>
		<td>0</td>
		<td>0</td>
		<td>1</td>
		<td>&nbsp;</td>
		<td>0</td>
		<td>1</td>
		<td>1</td>
		<td bgcolor="Grey95" border="1">0</td>
	</tr>
	<tr>
		<td align="left" port="1">Arcturus<font color="Grey75">\detokenize{⊔}</font></td>
		<td>0</td>
		<td>1</td>
		<td>0</td>
		<td bgcolor="Grey95" border="1">0</td>
		<td>0</td>
		<td>0</td>
		<td>0</td>
		<td>1</td>
		<td>&nbsp;</td>
		<td>0</td>
		<td>1</td>
		<td>1</td>
		<td bgcolor="Black" color="White" border="1"><font color="White">1</font></td>
	</tr>
	<tr>
		<td align="left" port="2">Sol<font color="Grey75">\detokenize{⊔}</font></td>
		<td>0</td>
		<td>1</td>
		<td>0</td>
		<td bgcolor="Black" color="White" border="1"><font color="White">1</font></td>
		<td>0</td>
		<td bgcolor="Grey95" border="1">0</td>
	</tr>
	<tr>
		<td align="left" port="3">Vega<font color="Grey75">\detokenize{⊔}</font></td>
		<td>0</td>
		<td>1</td>
		<td>0</td>
		<td bgcolor="Black" color="White" border="1"><font color="White">1</font></td>
		<td>0</td>
		<td bgcolor="Black" color="White" border="1"><font color="White">1</font></td>
	</tr>
</table>>];
}
	}
	\subcaptionbox{Memory.\label{star-0:mem}}{
\digraph[scale=0.5]{star0mem}{
	graph [truecolor=true, bgcolor=transparent, fontname=modern];
	node [shape=none, fontname=modern];
	tree0x100a04120branch0 [label = <
<table border="0" cellspacing="0">
	<tr><td colspan="3" align="left"><font color="Grey75">Vakgimbat</font> \detokenize{∑}bit=0</td></tr>
	<hr/>
	<tr>
		<td><font face="Times-Italic">left</font></td>
		<td><font face="Times-Italic">skip</font></td>
		<td><font face="Times-Italic">leaves</font></td>
	</tr>
	<hr/>
	<tr>
		<td align="right">1</td>
		<td align="right">3</td>
		<td align="left" port="0">Achernar<font color="Grey75">\detokenize{⊔}</font></td>
	</tr>
	<tr>
		<td align="right" bgcolor="Gray95">0</td>
		<td align="right" bgcolor="Gray95">7</td>
		<td align="left" port="1" bgcolor="Gray95">Arcturus<font color="Grey75">\detokenize{⊔}</font></td>
	</tr>
	<tr>
		<td align="right">0</td>
		<td align="right">1</td>
		<td align="left" port="2">Sol<font color="Grey75">\detokenize{⊔}</font></td>
	</tr>
	<tr>
		<td></td>
		<td></td>
		<td align="left" port="3" bgcolor="Gray95">Vega<font color="Grey75">\detokenize{⊔}</font></td>
	</tr>
	<hr/>
	<tr><td></td></tr>
</table>>];
}
	}
	\subcaptionbox{Tree.\label{star-0:tree}}{
\digraph[scale=\Scale]{star0tree}{
	graph [truecolor=true, bgcolor=transparent, fontname=modern];
	node [shape=none, fontname=modern];
	tree0x100a04120branch0 [label="3", shape=circle, style=filled, fillcolor=Grey95];
	tree0x100a04120branch0 -> tree0x100a04120branch1 [arrowhead=rnormal];
	tree0x100a04120branch0 -> tree0x100a04120branch2 [arrowhead=lnormal];
	tree0x100a04120branch1 [label="7", shape=circle, style=filled, fillcolor=Grey95];
	tree0x100a04120branch1 -> tree0x100a04120leaf0 [color=Gray75, arrowhead=rnormal];
	tree0x100a04120branch1 -> tree0x100a04120leaf1 [color=Gray75, arrowhead=lnormal];
	tree0x100a04120branch2 [label="1", shape=circle, style=filled, fillcolor=Grey95];
	tree0x100a04120branch2 -> tree0x100a04120leaf2 [color=Gray75, arrowhead=rnormal];
	tree0x100a04120branch2 -> tree0x100a04120leaf3 [color=Gray75, arrowhead=lnormal];
	tree0x100a04120leaf0 [label = <Achernar<font color="Gray75">\detokenize{⊔}</font>>];
	tree0x100a04120leaf1 [label = <Arcturus<font color="Gray75">\detokenize{⊔}</font>>];
	tree0x100a04120leaf2 [label = <Sol<font color="Gray75">\detokenize{⊔}</font>>];
	tree0x100a04120leaf3 [label = <Vega<font color="Gray75">\detokenize{⊔}</font>>];
}
	}
	\caption{A trie with three different views of the data.\label{star-0}}
\end{figure}

Figure~\ref{star-0:bits} is a visual example of a Patrica trie\cite{morrison1968patricia}, that is, a binary radix tree and skip values when bits offer no difference. Note that, in ASCII and UTF-8, \code{A} is represented by an octet with the value of 65, binary 01000001; \code{c} 99, 01100011; \code{r} 114, 01110010; \code{S} 83, 01010011; \code{V} 86, 01010110.

We encode the branches in pre-order fashion, as in Figure~\ref{star-0:mem}. Each branch has a \code{left} and a \code{skip}, corresponding to how many branches are descendants on the left, and how many bits we should skip before the decision bit. With the initial range set to the total number of branches, it becomes a matter of accumulating leaf values for the right branches of a key, accessing the index skip-sequentially, until the range is zero. The right values are implicit in the range. The leaves, on the other hand, are alphabetized, in-order. There will always be one less branch than leaf; that is, this is a full (strict) binary tree with $order - 1$ branches, for $order$ keys as leaves.

Figure~\ref{star-0:tree} shows the conventional full binary tree view of the same data as Figure~\ref{star-0:bits} and \ref{star-0:mem}. The branches indicate a \code{do not care} for all the skipped bits. If a query might have a difference in the skipped values, one can also check the final leaf for agreement with the found value.

\subsection{Range and locality}

Only when the algorithm arrives at a leaf will it go outside the \code{left-skip} index. This suggests that the index, and the size of the trie, be contiguous and as small as possible to fit the maximum into cache.

However, in establishing a maximum \code{skip} value, one limits the contiguous bits that can be skipped; this has an effect on both on insertion and deletion. More noticeably, the maximum \code{left} plus one is the maximum number of leaves in the worst-case of all-left. It is also inefficient to modify the trie with more and more keys; this requires more branches to be changed and an array insertion of the leaf.

To combat these two contradictory requirements, we have broken up the trie in much the same manner as \cite{bayer1972organization}. Except in tries, contrary to B-trees, the data can not be rotated at will; instead, it uses a bitmap of which leaves are links to other structures. We use that a trie is a forest of non-empty complete binary trees. A tree corresponds to a B-tree node\cite{knuth1997sorting}, that is, a contiguous area in memory. This would conflict with a key as a leaf and individual branches, which is are longer implicit.

Thus, on adding to a tree in a trie that has the maximum number of keys, we must split it into two trees. We use the fact that a binary tree of $n \ge 2$ nodes can be split into two trees not exceeding $\left\lceil \frac{2n-1}{3} \right\rceil$ nodes by starting \code{cut} at the root and choosing the subtree that is larger until the bound is achieved; the cut will be be between \code{cut} and it's parent. The parent will have an extra linking leaf.

\subsection{Example}

% star-4
\begin{figure}
	\centering
	\subcaptionbox{Bit view.\label{star-4:bits}}{
\digraph[scale=\Scale]{star4bits}{
	node [shape = none];
	tree0x1028049f0branch0 [shape = box, style = filled, fillcolor="Grey95" label = <
<TABLE BORDER="0" CELLBORDER="0">
	<TR>
		<TD ALIGN="LEFT" BORDER="0" PORT="0">\detokenize{↓}<FONT COLOR="Gray">Algieba</FONT></TD>
		<TD>0</TD>
		<TD>1</TD>
		<TD>0</TD>
		<TD BGCOLOR="White" BORDER="1">0</TD>
	</TR>
	<TR>
		<TD ALIGN="LEFT" BORDER="0" PORT="1">\detokenize{↓}<FONT COLOR="Gray">Regulus</FONT></TD>
		<TD>0</TD>
		<TD>1</TD>
		<TD>0</TD>
		<TD BGCOLOR="Black" COLOR="White" BORDER="1"><FONT COLOR="White">1</FONT></TD>
	</TR>
</TABLE>>];
	tree0x1028049f0branch0:0 -> tree0x1028049c0branch0 [color = "Black:invis:Black" style = dashed];
	tree0x1028049f0branch0:1 -> tree0x102804a20branch0 [color = "Black:invis:Black"];
	tree0x1028049c0branch0 [shape = box, style = filled, fillcolor="Grey95" label = <
<TABLE BORDER="0" CELLBORDER="0">
	<TR>
		<TD ALIGN="LEFT" BORDER="0" PORT="0">Algieba</TD>
		<TD>0</TD>
		<TD>1</TD>
		<TD>0</TD>
		<TD>0</TD>
		<TD>0</TD>
		<TD>0</TD>
		<TD>0</TD>
		<TD>1</TD>
		<TD BORDER="0">&nbsp;</TD>
		<TD>0</TD>
		<TD>1</TD>
		<TD>1</TD>
		<TD>0</TD>
		<TD>1</TD>
		<TD>1</TD>
		<TD>0</TD>
		<TD>0</TD>
		<TD BORDER="0">&nbsp;</TD>
		<TD>0</TD>
		<TD>1</TD>
		<TD>1</TD>
		<TD BGCOLOR="White" BORDER="1">0</TD>
	</TR>
	<TR>
		<TD ALIGN="LEFT" BORDER="0" PORT="1">Alpheratz</TD>
		<TD>0</TD>
		<TD>1</TD>
		<TD>0</TD>
		<TD>0</TD>
		<TD>0</TD>
		<TD>0</TD>
		<TD>0</TD>
		<TD>1</TD>
		<TD BORDER="0">&nbsp;</TD>
		<TD>0</TD>
		<TD>1</TD>
		<TD>1</TD>
		<TD>0</TD>
		<TD>1</TD>
		<TD>1</TD>
		<TD>0</TD>
		<TD>0</TD>
		<TD BORDER="0">&nbsp;</TD>
		<TD>0</TD>
		<TD>1</TD>
		<TD>1</TD>
		<TD BGCOLOR="Black" COLOR="White" BORDER="1"><FONT COLOR="White">1</FONT></TD>
	</TR>
</TABLE>>];
	tree0x102804a20branch0 [shape = box, style = filled, fillcolor="Grey95" label = <
<TABLE BORDER="0" CELLBORDER="0">
	<TR>
		<TD ALIGN="LEFT" BORDER="0" PORT="0">Regulus</TD>
		<TD>0</TD>
		<TD>1</TD>
		<TD>0</TD>
		<TD>1</TD>
		<TD>0</TD>
		<TD BGCOLOR="White" BORDER="1">0</TD>
	</TR>
	<TR>
		<TD ALIGN="LEFT" BORDER="0" PORT="1">Vega</TD>
		<TD>0</TD>
		<TD>1</TD>
		<TD>0</TD>
		<TD>1</TD>
		<TD>0</TD>
		<TD BGCOLOR="Black" COLOR="White" BORDER="1"><FONT COLOR="White">1</FONT></TD>
	</TR>
</TABLE>>];
}}
	\subcaptionbox{Logical view.\label{star-4:logic}}{
\digraph[scale=\Scale]{star4logic}{
	node [shape = none];
	tree0x1028049f0branch0 [label = "3", shape = circle, style = filled, fillcolor = Grey95];
	tree0x1028049f0branch0 -> tree0x1028049c0branch0 [style = dashed, color = "Black:invis:Black"];
	tree0x1028049f0branch0 -> tree0x102804a20branch0 [color = "Black:invis:Black"];
	tree0x1028049c0branch0 [label = "15", shape = circle, style = filled, fillcolor = Grey95];
	tree0x1028049c0branch0 -> tree0x1028049c0leaf0 [style = dashed, color = Gray];
	tree0x1028049c0branch0 -> tree0x1028049c0leaf1 [color = Gray];
	tree0x1028049c0leaf0 [label = "Algieba"];
	tree0x1028049c0leaf1 [label = "Alpheratz"];
	tree0x102804a20branch0 [label = "1", shape = circle, style = filled, fillcolor = Grey95];
	tree0x102804a20branch0 -> tree0x102804a20leaf0 [style = dashed, color = Gray];
	tree0x102804a20branch0 -> tree0x102804a20leaf1 [color = Gray];
	tree0x102804a20leaf0 [label = "Regulus"];
	tree0x102804a20leaf1 [label = "Vega"];
}}\\
	\subcaptionbox{Memory view.\label{star-4:mem}}{
\digraph[scale=\Scale]{star4mem}{
	node [shape = none];
	tree0x1028049f0branch0 [shape = box, style = filled, fillcolor = Gray95, label = <
<TABLE BORDER="0">
	<TR>
		<TD ALIGN="right" BORDER="0">left</TD>
		<TD BGCOLOR="Gray90">0</TD>
	</TR>
	<TR>
		<TD ALIGN="right" BORDER="0">skip</TD>
		<TD>3</TD>
	</TR>
	<TR>
		<TD ALIGN="right" BORDER="0">leaves</TD>
		<TD PORT="0" BGCOLOR="Gray90">...</TD>
		<TD PORT="1" BGCOLOR="Gray90">...</TD>
	</TR>
</TABLE>>];
	tree0x1028049f0branch0:0 -> tree0x1028049c0branch0 [color = "Black:invis:Black" style = dashed];
	tree0x1028049f0branch0:1 -> tree0x102804a20branch0 [color = "Black:invis:Black"];
	tree0x1028049c0branch0 [shape = box, style = filled, fillcolor = Gray95, label = <
<TABLE BORDER="0">
	<TR>
		<TD ALIGN="right" BORDER="0">left</TD>
		<TD BGCOLOR="Gray90">0</TD>
	</TR>
	<TR>
		<TD ALIGN="right" BORDER="0">skip</TD>
		<TD>15</TD>
	</TR>
	<TR>
		<TD ALIGN="right" BORDER="0">leaves</TD>
		<TD BGCOLOR="Grey90">Algieba</TD>
		<TD BGCOLOR="Grey90">Alpheratz</TD>
	</TR>
</TABLE>>];
	tree0x102804a20branch0 [shape = box, style = filled, fillcolor = Gray95, label = <
<TABLE BORDER="0">
	<TR>
		<TD ALIGN="right" BORDER="0">left</TD>
		<TD BGCOLOR="Gray90">0</TD>
	</TR>
	<TR>
		<TD ALIGN="right" BORDER="0">skip</TD>
		<TD>1</TD>
	</TR>
	<TR>
		<TD ALIGN="right" BORDER="0">leaves</TD>
		<TD BGCOLOR="Grey90">Regulus</TD>
		<TD BGCOLOR="Grey90">Vega</TD>
	</TR>
</TABLE>>];
}}
	\caption{Split the 3-trie to put four items.\label{star-4}}
\end{figure}



\subsection{Hysteresis}

Issues with having zero, one, zero, one, zero, items always needing to allocate memory. Fixme: store the order instead of the branches.

\subsection{Machine Considerations}

The data has been engineered for maximum effectiveness of the cache in reading and traversing. That is, the tree structure and the string decisions have been reduced to each a byte and placed at the the top of the data structure of the tree. Always forward in memory. The size of this sub-structure should be a multiple of the cache line size, while also maximizing the dynamic range of $\mathit{left}$; a trie (also a B-tree) of order 256 is an obvious choice.\cite{sinha2004cache}

\subsection{Running Time}

\cite{shannon1948mathematical} $\log n == m$.

The `\O(log n)` running time is only when the trie has bounds on strings that are placed there. Worst-case $\{ a, aa, aaa, aaaa, ... \}$. The case where the trie has strings that are bounded, as the tree grows, we can guarantee . . .

\subsection{Limits}

The skip value is limited by its range; in this case, 255 bits. For example, this trie is valid, $\{ dictionary \}$, as well as, $\{ dictator, dictionary, dictionaries \}$, but one ca'n't transition to, $\{ dictionary, dictionaries \}$, because it is too long a skip value. There are several modifications that would allow this, but they are out of this scope. (This is not true; 8 x 8 = 64; 8 x 32 = 256.)

Insertion: Any leaf on the sub-tree queried will do; in this implementation, favours the left side.

on split, do we have to go locally and see if we can join them?

we don't need to store any data if the leaf-trees are different from the branch-trees?
$
trie: root, height
branch tree: bsize, +branchtree, branch[o-1], leaf[o]
leaf tree: bsize, ?, branch[o-1], is_recursive_bmp[o/8], leaf[o]
or
leaf tree: bsize, ?, branch[o-1], leaf[o] { skip, union{ data, trie } }
or
leaf tree: bsize, ?, branch[o-1], leaf[o] { 32:skip, 32:height, 64:root / 64:data }
$

\bibliography{trie}

\end{document}
