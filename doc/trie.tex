\documentclass[12pt]{article}

% input
\bibliographystyle{ieeetr}
\usepackage[utf8]{inputenc}
%\usepackage{times} % font
\usepackage{lmodern} % scalable font
\usepackage[pdftex]{graphicx} % include graphics
\usepackage{amsmath} % align, nobreakdash
\usepackage[pdf,tmpdir]{graphviz} % digraph
\usepackage{fullpage} % book margins -> std margins
\usepackage{wrapfig} % wrapfigure
%\usepackage{moreverb} % verbatimtabinput
\usepackage{subcaption} % subcaptionbox
\usepackage[colorlinks]{hyperref} % pdf links
\usepackage{url} % url support
%\usepackage{comment} % comment

\usepackage{siunitx}
%\usepackage{graphicx} already
\usepackage{latexsym}
\usepackage{keyval}
\usepackage{ifthen}
\usepackage{moreverb}
\usepackage[siunitx, subfolder]{gnuplottex}

% code doesn't wrap
\usepackage[table]{xcolor}
\definecolor{light-gray}{gray}{0.95}
\newcommand{\code}[1]{\colorbox{light-gray}{\texttt{#1}}}

% create new commands
%\def\^#1{\textsuperscript{#1}}
%\def\!{\overline}
%\def\degree{\ensuremath{^\circ}}
\def\Scale{0.5}

% colourize titles
\definecolor{ilrblue}{RGB}{79,166,220}
\usepackage{titling}
\pretitle{\vspace{-3em}\fontfamily{\sfdefault}\fontsize{18bp}{18bp}\color{ilrblue}\selectfont}
\posttitle{\par\vspace{18bp}}
\preauthor{\normalfont\bfseries\selectfont\MakeUppercase}
\postauthor{\par\vspace{4bp}}
\predate{\normalfont\selectfont}
\postdate{\par\vspace{-8bp}}
\usepackage{titlesec}
\titleformat{\section}{\fontfamily{\sfdefault}\selectfont\normalsize\bfseries\color{ilrblue}\MakeUppercase}{\thesection}{1em}{}
\titleformat{\subsection}{\fontfamily{\sfdefault}\normalsize\bfseries\color{ilrblue}}{\thesubsection}{1em}{}
\titleformat{\subsubsection}{\fontfamily{\sfdefault}\normalsize\bfseries\color{ilrblue}\it}{\thesubsubsection}{1em}{}
% Ewww
\makeatletter
\renewenvironment{abstract}{%
    \if@twocolumn
      \section*{\abstractname}%
    \else \small %
      \begin{center}%
        {\bfseries\color{ilrblue} \abstractname\vspace{\z@}\vspace{-8bp}}%
      \end{center}%
      \quotation
    \fi}
    {\if@twocolumn\else\endquotation\fi}
\makeatother

% for hyperref
\hypersetup{
  linkcolor=ilrblue, % internal (figure) links
  urlcolor=ilrblue,
  filecolor=ilrblue,
  citecolor=ilrblue, % bibliography links
  pdfauthor={\@author},
  pdftitle={\@title},
  pdfsubject={\@title},
  pdfpagemode=UseNone
}

\author{Neil A. Edelman}
\title{Compact binary prefix trees}
\date{2021-10-20}

\begin{document}

\maketitle

\abstract{Our prefix-tree, digital-tree, or trie is an ordered set or map with key strings. We build a dynamic index of two-bytes {\it per} entry, only storing differences in a compact binary radix tree. To maximize locality of reference while descending the trie and minimizing update data, these are grouped together in a forest of fix-sized trees. In practice, this trie is comparable to a B-tree in performance.}

\section{Introduction}

A trie is a tree that stores partitioned sets of strings\cite{de1959file, fredkin1960trie, jacquet1991analysis, askitis2011redesigning} so that, ``instead of basing a search method on comparisons between keys, we can make use of their representation as a sequence of digits or alphabetic characters [directly].\cite{knuth1997sorting}'' It is necessarily ordered, and allows prefix range queries.

Often, only parts of the key string are important; a radix trie (compact prefix tree) skips past the parts that are not important, as \cite{askitis2007hat}. If a candidate key match is found, a full match can be made with one index from the trie.

For most applications, a 256-ary trie is space-intensive; the index contains many spaces for keys that are unused. Compression schemes are available, such as re-using a pool of memory\cite{de1959file}, reducing our encoding alphabet, or take smaller than 8-bit chunks\cite{fredkin1960trie}.

We use a combination binary radix trie, described in \cite{morrison1968patricia} as the PATRICIA automaton. Rather than being sparse, a Patricia-trie is a packed index. It is sometimes convenient to think of this as a full binary tree whose branches store the number of skip bits before the cursor, and splits according to 0 or 1 of the decision bit. The leaves, therefore, are keys, and any other information associated with the key, necessarily corresponding to the path through the branches. Examples of this are seen in Figure~\ref{star-0:tree} and \ref{star-1:tree}.

\section{Implementation}

\subsection{Encoding}

\begin{figure}
	\centering
	\subcaptionbox{bits\label{star-0:bits}}{%
\digraph[scale=0.6]{star0bits}{
	graph [truecolor=true, bgcolor=transparent, fontname=modern];
	node [shape=none, fontname=modern];
	tree0x100a04120branch0 [label = <
<table border="0" cellspacing="0">
	<tr>
		<td align="left" port="0">Achernar<font color="Grey75">\detokenize{⊔}</font></td>
		<td>0</td>
		<td>1</td>
		<td>0</td>
		<td bgcolor="Grey95" border="1">0</td>
		<td>0</td>
		<td>0</td>
		<td>0</td>
		<td>1</td>
		<td>&nbsp;</td>
		<td>0</td>
		<td>1</td>
		<td>1</td>
		<td bgcolor="Grey95" border="1">0</td>
	</tr>
	<tr>
		<td align="left" port="1">Arcturus<font color="Grey75">\detokenize{⊔}</font></td>
		<td>0</td>
		<td>1</td>
		<td>0</td>
		<td bgcolor="Grey95" border="1">0</td>
		<td>0</td>
		<td>0</td>
		<td>0</td>
		<td>1</td>
		<td>&nbsp;</td>
		<td>0</td>
		<td>1</td>
		<td>1</td>
		<td bgcolor="Black" color="White" border="1"><font color="White">1</font></td>
	</tr>
	<tr>
		<td align="left" port="2">Sol<font color="Grey75">\detokenize{⊔}</font></td>
		<td>0</td>
		<td>1</td>
		<td>0</td>
		<td bgcolor="Black" color="White" border="1"><font color="White">1</font></td>
		<td>0</td>
		<td bgcolor="Grey95" border="1">0</td>
	</tr>
	<tr>
		<td align="left" port="3">Vega<font color="Grey75">\detokenize{⊔}</font></td>
		<td>0</td>
		<td>1</td>
		<td>0</td>
		<td bgcolor="Black" color="White" border="1"><font color="White">1</font></td>
		<td>0</td>
		<td bgcolor="Black" color="White" border="1"><font color="White">1</font></td>
	</tr>
</table>>];
}
	}
	\subcaptionbox{memory\label{star-0:mem}}{
\digraph[scale=0.5]{star0mem}{
	graph [truecolor=true, bgcolor=transparent, fontname=modern];
	node [shape=none, fontname=modern];
	tree0x100a04120branch0 [label = <
<table border="0" cellspacing="0">
	<tr><td colspan="3" align="left"><font color="Grey75">Vakgimbat</font> \detokenize{∑}bit=0</td></tr>
	<hr/>
	<tr>
		<td><font face="Times-Italic">left</font></td>
		<td><font face="Times-Italic">skip</font></td>
		<td><font face="Times-Italic">leaves</font></td>
	</tr>
	<hr/>
	<tr>
		<td align="right">1</td>
		<td align="right">3</td>
		<td align="left" port="0">Achernar<font color="Grey75">\detokenize{⊔}</font></td>
	</tr>
	<tr>
		<td align="right" bgcolor="Gray95">0</td>
		<td align="right" bgcolor="Gray95">7</td>
		<td align="left" port="1" bgcolor="Gray95">Arcturus<font color="Grey75">\detokenize{⊔}</font></td>
	</tr>
	<tr>
		<td align="right">0</td>
		<td align="right">1</td>
		<td align="left" port="2">Sol<font color="Grey75">\detokenize{⊔}</font></td>
	</tr>
	<tr>
		<td></td>
		<td></td>
		<td align="left" port="3" bgcolor="Gray95">Vega<font color="Grey75">\detokenize{⊔}</font></td>
	</tr>
	<hr/>
	<tr><td></td></tr>
</table>>];
}
	}
	\subcaptionbox{tree\label{star-0:tree}}{
\digraph[scale=\Scale]{star0tree}{
	graph [truecolor=true, bgcolor=transparent, fontname=modern];
	node [shape=none, fontname=modern];
	tree0x100a04120branch0 [label="3", shape=circle, style=filled, fillcolor=Grey95];
	tree0x100a04120branch0 -> tree0x100a04120branch1 [arrowhead=rnormal];
	tree0x100a04120branch0 -> tree0x100a04120branch2 [arrowhead=lnormal];
	tree0x100a04120branch1 [label="7", shape=circle, style=filled, fillcolor=Grey95];
	tree0x100a04120branch1 -> tree0x100a04120leaf0 [color=Gray75, arrowhead=rnormal];
	tree0x100a04120branch1 -> tree0x100a04120leaf1 [color=Gray75, arrowhead=lnormal];
	tree0x100a04120branch2 [label="1", shape=circle, style=filled, fillcolor=Grey95];
	tree0x100a04120branch2 -> tree0x100a04120leaf2 [color=Gray75, arrowhead=rnormal];
	tree0x100a04120branch2 -> tree0x100a04120leaf3 [color=Gray75, arrowhead=lnormal];
	tree0x100a04120leaf0 [label = <Achernar<font color="Gray75">\detokenize{⊔}</font>>];
	tree0x100a04120leaf1 [label = <Arcturus<font color="Gray75">\detokenize{⊔}</font>>];
	tree0x100a04120leaf2 [label = <Sol<font color="Gray75">\detokenize{⊔}</font>>];
	tree0x100a04120leaf3 [label = <Vega<font color="Gray75">\detokenize{⊔}</font>>];
}
	}
	\caption{A trie with three different views of the data.\label{star-0}}
\end{figure}

In practice, we talk about a string always terminated by a sentinel; this is an easy way to allow a string and it's prefix in the same trie\cite{morrison1968patricia}. In C, a NUL-terminated string automatically has this property, and is ordered correctly. Keys are sorted in lexicographic order by numerical value; \code{strcmp}-order, not by any collation algorithm.

Figure~\ref{star-0:bits} is a visual example of a Patricia trie, that is, a binary radix tree and skip values when bits offer no difference. Note that, in ASCII and UTF-8, \code{A} is represented by an octet with the value of 65, binary 01000001; \code{c} 99, 01100011; \code{r} 114, 01110010; \code{S} 83, 01010011; \code{V} 86, 01010110.

We encode the branches in pre-order fashion, as in Figure~\ref{star-0:mem}. Each branch has a \code{left} and a \code{skip}, corresponding to how many branches are descendants on the left, and how many bits we should skip before the decision bit. With the initial range set to the total number of branches, it becomes a matter of accumulating leaf values for the right branches of a key, accessing the index skip-sequentially, until the range is zero. The right values are implicit in the range. The leaves, on the other hand, are alphabetized, in-order. There will always be one less branch than leaf; that is, this is a full (strict) binary tree with $order - 1$ branches, for $order$ keys as leaves.

Figure~\ref{star-0:tree} shows the conventional full binary tree view of the same data as Figure~\ref{star-0:bits} and \ref{star-0:mem}. The branches indicate a \code{do not care} for all the skipped bits. If a query might have a difference in the skipped values, one can also check the final leaf for agreement with the found value.

\subsection{Range and locality}

Only when the algorithm arrives at a leaf will it go outside the \code{left, skip}. This suggests that these be placed in a contiguous index. This index should be compact as possible to fit the maximum into cache.

However, in establishing a maximum \code{skip} value, one limits the contiguous bits that can be skipped; this has an effect on both on insertion and deletion. One octet provides 255 bits skip, usually enough for approximately 32 bytes. More noticeably, the maximum \code{left} plus one is the maximum number of leaves in the worst-case of all-left. It is also inefficient to modify the trie with more and more keys; this requires more branches to be changed and an array insertion of the leaf.

To combat these two contradictory requirements, we have broken up the trie in much the same manner as \cite{bayer1972organization}. Except in tries, contrary to B-trees, the data can not be rotated at will; instead, it relaxes the rules and instead uses a bitmap of which leaves are links to other structures, called trees. Thus a trie is a forest of non-empty full binary trees. A tree corresponds to a B-tree node\cite{knuth1997sorting}, that is, a contiguous area in memory. This would conflict with the terminology of a key as a leaf and individual branches, which are longer implicit.

Thus, on adding to a tree in a trie that has the maximum number of keys, we must split it into two trees. We use the fact that a binary tree of $n \ge 2$ nodes can be split into two trees not exceeding $\left\lceil \frac{2n-1}{3} \right\rceil$ nodes by starting \code{daughter} tree at the root and choosing the subtree that is larger until the bound is achieved. The \code{mother} will have an extra linking leaf.

\subsection{Link keys}

A more complex example is given in Figure~\ref{star-1}. This trie has 3 fixed trees of order 7 maximum leaves and 6 maximum branches with 14 keys in total.

\begin{figure}
	\centering
	\subcaptionbox{bits\label{star-1:bits}}{
\digraph[scale=\Scale]{star1bits}{
	graph [truecolor=true, bgcolor=transparent, fontname=modern];
	node [shape=none, fontname=modern];
	tree0x100a04120branch0 [label = <
<table border="0" cellspacing="0">
	<tr>
		<td align="left" port="0">\detokenize{↓}<font color="Grey75">Altair\detokenize{⊔}</font></td>
		<td>0</td>
		<td>1</td>
		<td>0</td>
		<td bgcolor="Grey95" border="1">0</td>
		<td>0</td>
		<td bgcolor="Grey95" border="1">0</td>
	</tr>
	<tr>
		<td align="left" port="1">Fomalhaut<font color="Grey75">\detokenize{⊔}</font></td>
		<td>0</td>
		<td>1</td>
		<td>0</td>
		<td bgcolor="Grey95" border="1">0</td>
		<td>0</td>
		<td bgcolor="Black" color="White" border="1"><font color="White">1</font></td>
		<td>1</td>
		<td bgcolor="Grey95" border="1">0</td>
	</tr>
	<tr>
		<td align="left" port="2">Gacrux<font color="Grey75">\detokenize{⊔}</font></td>
		<td>0</td>
		<td>1</td>
		<td>0</td>
		<td bgcolor="Grey95" border="1">0</td>
		<td>0</td>
		<td bgcolor="Black" color="White" border="1"><font color="White">1</font></td>
		<td>1</td>
		<td bgcolor="Black" color="White" border="1"><font color="White">1</font></td>
	</tr>
	<tr>
		<td align="left" port="3">\detokenize{↓}<font color="Grey75">Polaris\detokenize{⊔}</font></td>
		<td>0</td>
		<td>1</td>
		<td>0</td>
		<td bgcolor="Black" color="White" border="1"><font color="White">1</font></td>
	</tr>
</table>>];
	tree0x100a04120branch0:0 -> tree0x100b04160branch0 [style = dashed, arrowhead = rnormal];
	tree0x100a04120branch0:3 -> tree0x100b040d0branch0 [style = dashed, arrowhead = lnormal];
	tree0x100b04160branch0 [label = <
<table border="0" cellspacing="0">
	<tr>
		<td align="left" port="0">Altair<font color="Grey75">\detokenize{⊔}</font></td>
		<td>0</td>
		<td>1</td>
		<td>0</td>
		<td>0</td>
		<td>0</td>
		<td>0</td>
		<td bgcolor="Grey95" border="1">0</td>
	</tr>
	<tr>
		<td align="left" port="1">Betelgeuse<font color="Grey75">\detokenize{⊔}</font></td>
		<td>0</td>
		<td>1</td>
		<td>0</td>
		<td>0</td>
		<td>0</td>
		<td>0</td>
		<td bgcolor="Black" color="White" border="1"><font color="White">1</font></td>
		<td bgcolor="Grey95" border="1">0</td>
	</tr>
	<tr>
		<td align="left" port="2">Canopus<font color="Grey75">\detokenize{⊔}</font></td>
		<td>0</td>
		<td>1</td>
		<td>0</td>
		<td>0</td>
		<td>0</td>
		<td>0</td>
		<td bgcolor="Black" color="White" border="1"><font color="White">1</font></td>
		<td bgcolor="Black" color="White" border="1"><font color="White">1</font></td>
		<td>&nbsp;</td>
		<td>0</td>
		<td>1</td>
		<td>1</td>
		<td>0</td>
		<td>0</td>
		<td>0</td>
		<td>0</td>
		<td>1</td>
		<td>&nbsp;</td>
		<td>0</td>
		<td>1</td>
		<td>1</td>
		<td bgcolor="Grey95" border="1">0</td>
	</tr>
	<tr>
		<td align="left" port="3">Capella<font color="Grey75">\detokenize{⊔}</font></td>
		<td>0</td>
		<td>1</td>
		<td>0</td>
		<td>0</td>
		<td>0</td>
		<td>0</td>
		<td bgcolor="Black" color="White" border="1"><font color="White">1</font></td>
		<td bgcolor="Black" color="White" border="1"><font color="White">1</font></td>
		<td>&nbsp;</td>
		<td>0</td>
		<td>1</td>
		<td>1</td>
		<td>0</td>
		<td>0</td>
		<td>0</td>
		<td>0</td>
		<td>1</td>
		<td>&nbsp;</td>
		<td>0</td>
		<td>1</td>
		<td>1</td>
		<td bgcolor="Black" color="White" border="1"><font color="White">1</font></td>
		<td>0</td>
		<td>0</td>
		<td bgcolor="Grey95" border="1">0</td>
	</tr>
	<tr>
		<td align="left" port="4">Castor<font color="Grey75">\detokenize{⊔}</font></td>
		<td>0</td>
		<td>1</td>
		<td>0</td>
		<td>0</td>
		<td>0</td>
		<td>0</td>
		<td bgcolor="Black" color="White" border="1"><font color="White">1</font></td>
		<td bgcolor="Black" color="White" border="1"><font color="White">1</font></td>
		<td>&nbsp;</td>
		<td>0</td>
		<td>1</td>
		<td>1</td>
		<td>0</td>
		<td>0</td>
		<td>0</td>
		<td>0</td>
		<td>1</td>
		<td>&nbsp;</td>
		<td>0</td>
		<td>1</td>
		<td>1</td>
		<td bgcolor="Black" color="White" border="1"><font color="White">1</font></td>
		<td>0</td>
		<td>0</td>
		<td bgcolor="Black" color="White" border="1"><font color="White">1</font></td>
	</tr>
</table>>];
	tree0x100b040d0branch0 [label = <
<table border="0" cellspacing="0">
	<tr>
		<td align="left" port="0">Polaris<font color="Grey75">\detokenize{⊔}</font></td>
		<td>0</td>
		<td>1</td>
		<td>0</td>
		<td>1</td>
		<td>0</td>
		<td bgcolor="Grey95" border="1">0</td>
		<td bgcolor="Grey95" border="1">0</td>
		<td>0</td>
		<td>&nbsp;</td>
		<td>0</td>
		<td>1</td>
		<td>1</td>
		<td>0</td>
		<td>1</td>
		<td>1</td>
		<td>1</td>
		<td>1</td>
		<td>&nbsp;</td>
		<td>0</td>
		<td>1</td>
		<td>1</td>
		<td>0</td>
		<td>1</td>
		<td>1</td>
		<td>0</td>
		<td>0</td>
		<td>&nbsp;</td>
		<td>0</td>
		<td>1</td>
		<td>1</td>
		<td>0</td>
		<td bgcolor="Grey95" border="1">0</td>
	</tr>
	<tr>
		<td align="left" port="1">Pollux<font color="Grey75">\detokenize{⊔}</font></td>
		<td>0</td>
		<td>1</td>
		<td>0</td>
		<td>1</td>
		<td>0</td>
		<td bgcolor="Grey95" border="1">0</td>
		<td bgcolor="Grey95" border="1">0</td>
		<td>0</td>
		<td>&nbsp;</td>
		<td>0</td>
		<td>1</td>
		<td>1</td>
		<td>0</td>
		<td>1</td>
		<td>1</td>
		<td>1</td>
		<td>1</td>
		<td>&nbsp;</td>
		<td>0</td>
		<td>1</td>
		<td>1</td>
		<td>0</td>
		<td>1</td>
		<td>1</td>
		<td>0</td>
		<td>0</td>
		<td>&nbsp;</td>
		<td>0</td>
		<td>1</td>
		<td>1</td>
		<td>0</td>
		<td bgcolor="Black" color="White" border="1"><font color="White">1</font></td>
	</tr>
	<tr>
		<td align="left" port="2">Regulus<font color="Grey75">\detokenize{⊔}</font></td>
		<td>0</td>
		<td>1</td>
		<td>0</td>
		<td>1</td>
		<td>0</td>
		<td bgcolor="Grey95" border="1">0</td>
		<td bgcolor="Black" color="White" border="1"><font color="White">1</font></td>
		<td bgcolor="Grey95" border="1">0</td>
		<td>&nbsp;</td>
		<td>0</td>
		<td>1</td>
		<td>1</td>
		<td>0</td>
		<td bgcolor="Grey95" border="1">0</td>
	</tr>
	<tr>
		<td align="left" port="3">Rigel<font color="Grey75">\detokenize{⊔}</font></td>
		<td>0</td>
		<td>1</td>
		<td>0</td>
		<td>1</td>
		<td>0</td>
		<td bgcolor="Grey95" border="1">0</td>
		<td bgcolor="Black" color="White" border="1"><font color="White">1</font></td>
		<td bgcolor="Grey95" border="1">0</td>
		<td>&nbsp;</td>
		<td>0</td>
		<td>1</td>
		<td>1</td>
		<td>0</td>
		<td bgcolor="Black" color="White" border="1"><font color="White">1</font></td>
	</tr>
	<tr>
		<td align="left" port="4">Sirius<font color="Grey75">\detokenize{⊔}</font></td>
		<td>0</td>
		<td>1</td>
		<td>0</td>
		<td>1</td>
		<td>0</td>
		<td bgcolor="Grey95" border="1">0</td>
		<td bgcolor="Black" color="White" border="1"><font color="White">1</font></td>
		<td bgcolor="Black" color="White" border="1"><font color="White">1</font></td>
		<td>&nbsp;</td>
		<td>0</td>
		<td>1</td>
		<td>1</td>
		<td bgcolor="Grey95" border="1">0</td>
	</tr>
	<tr>
		<td align="left" port="5">Spica<font color="Grey75">\detokenize{⊔}</font></td>
		<td>0</td>
		<td>1</td>
		<td>0</td>
		<td>1</td>
		<td>0</td>
		<td bgcolor="Grey95" border="1">0</td>
		<td bgcolor="Black" color="White" border="1"><font color="White">1</font></td>
		<td bgcolor="Black" color="White" border="1"><font color="White">1</font></td>
		<td>&nbsp;</td>
		<td>0</td>
		<td>1</td>
		<td>1</td>
		<td bgcolor="Black" color="White" border="1"><font color="White">1</font></td>
	</tr>
	<tr>
		<td align="left" port="6">Vega<font color="Grey75">\detokenize{⊔}</font></td>
		<td>0</td>
		<td>1</td>
		<td>0</td>
		<td>1</td>
		<td>0</td>
		<td bgcolor="Black" color="White" border="1"><font color="White">1</font></td>
	</tr>
</table>>];
}
	}
	\subcaptionbox{memory\label{star-1:mem}}{
\digraph[scale=0.42]{star1mem}{
	graph [truecolor=true, bgcolor=transparent, fontname=modern];
	node [shape=none, fontname=modern];
	tree0x100a04120branch0 [label = <
<table border="0" cellspacing="0">
	<tr><td colspan="3" align="left"><font color="Grey75">Vakgimbat</font> \detokenize{∑}bit=0</td></tr>
	<hr/>
	<tr>
		<td><font face="Times-Italic">left</font></td>
		<td><font face="Times-Italic">skip</font></td>
		<td><font face="Times-Italic">leaves</font></td>
	</tr>
	<hr/>
	<tr>
		<td align="right">2</td>
		<td align="right">3</td>
		<td align="left" port="0">\detokenize{↓}<font color="Grey75">Altair\detokenize{⊔}</font></td>
	</tr>
	<tr>
		<td align="right" bgcolor="Gray95">0</td>
		<td align="right" bgcolor="Gray95">1</td>
		<td align="left" port="1" bgcolor="Gray95">Fomalhaut<font color="Grey75">\detokenize{⊔}</font></td>
	</tr>
	<tr>
		<td align="right">0</td>
		<td align="right">1</td>
		<td align="left" port="2">Gacrux<font color="Grey75">\detokenize{⊔}</font></td>
	</tr>
	<tr>
		<td></td>
		<td></td>
		<td align="left" port="3" bgcolor="Gray95">\detokenize{↓}<font color="Grey75">Polaris\detokenize{⊔}</font></td>
	</tr>
	<hr/>
	<tr><td></td></tr>
</table>>];
	tree0x100a04120branch0:0 -> tree0x100b04160branch0 [style = dashed, arrowhead = rnormal];
	tree0x100a04120branch0:3 -> tree0x100b040d0branch0 [style = dashed, arrowhead = lnormal];
	tree0x100b04160branch0 [label = <
<table border="0" cellspacing="0">
	<tr><td colspan="3" align="left"><font color="Grey75">Ukgul</font> \detokenize{∑}bit=6</td></tr>
	<hr/>
	<tr>
		<td><font face="Times-Italic">left</font></td>
		<td><font face="Times-Italic">skip</font></td>
		<td><font face="Times-Italic">leaves</font></td>
	</tr>
	<hr/>
	<tr>
		<td align="right">0</td>
		<td align="right">0</td>
		<td align="left" port="0">Altair<font color="Grey75">\detokenize{⊔}</font></td>
	</tr>
	<tr>
		<td align="right" bgcolor="Gray95">0</td>
		<td align="right" bgcolor="Gray95">0</td>
		<td align="left" port="1" bgcolor="Gray95">Betelgeuse<font color="Grey75">\detokenize{⊔}</font></td>
	</tr>
	<tr>
		<td align="right">0</td>
		<td align="right">11</td>
		<td align="left" port="2">Canopus<font color="Grey75">\detokenize{⊔}</font></td>
	</tr>
	<tr>
		<td align="right" bgcolor="Gray95">0</td>
		<td align="right" bgcolor="Gray95">2</td>
		<td align="left" port="3" bgcolor="Gray95">Capella<font color="Grey75">\detokenize{⊔}</font></td>
	</tr>
	<tr>
		<td></td>
		<td></td>
		<td align="left" port="4">Castor<font color="Grey75">\detokenize{⊔}</font></td>
	</tr>
	<hr/>
	<tr><td></td></tr>
</table>>];
	tree0x100b040d0branch0 [label = <
<table border="0" cellspacing="0">
	<tr><td colspan="3" align="left"><font color="Grey75">Orglob</font> \detokenize{∑}bit=4</td></tr>
	<hr/>
	<tr>
		<td><font face="Times-Italic">left</font></td>
		<td><font face="Times-Italic">skip</font></td>
		<td><font face="Times-Italic">leaves</font></td>
	</tr>
	<hr/>
	<tr>
		<td align="right">5</td>
		<td align="right">1</td>
		<td align="left" port="0">Polaris<font color="Grey75">\detokenize{⊔}</font></td>
	</tr>
	<tr>
		<td align="right" bgcolor="Gray95">1</td>
		<td align="right" bgcolor="Gray95">0</td>
		<td align="left" port="1" bgcolor="Gray95">Pollux<font color="Grey75">\detokenize{⊔}</font></td>
	</tr>
	<tr>
		<td align="right">0</td>
		<td align="right">21</td>
		<td align="left" port="2">Regulus<font color="Grey75">\detokenize{⊔}</font></td>
	</tr>
	<tr>
		<td align="right" bgcolor="Gray95">1</td>
		<td align="right" bgcolor="Gray95">0</td>
		<td align="left" port="3" bgcolor="Gray95">Rigel<font color="Grey75">\detokenize{⊔}</font></td>
	</tr>
	<tr>
		<td align="right">0</td>
		<td align="right">4</td>
		<td align="left" port="4">Sirius<font color="Grey75">\detokenize{⊔}</font></td>
	</tr>
	<tr>
		<td align="right" bgcolor="Gray95">0</td>
		<td align="right" bgcolor="Gray95">3</td>
		<td align="left" port="5" bgcolor="Gray95">Spica<font color="Grey75">\detokenize{⊔}</font></td>
	</tr>
	<tr>
		<td></td>
		<td></td>
		<td align="left" port="6">Vega<font color="Grey75">\detokenize{⊔}</font></td>
	</tr>
	<hr/>
	<tr><td></td></tr>
</table>>];
}
	}
	\subcaptionbox{tree\label{star-1:tree}}{
\digraph[scale=0.36]{star1tree}{
	graph [truecolor=true, bgcolor=transparent, fontname=modern];
	node [shape=none, fontname=modern];
	tree0x100a04120branch0 [label="3", shape=circle, style=filled, fillcolor="Grey95"];
	tree0x100a04120branch0 -> tree0x100a04120branch1 [arrowhead=rnormal];
	tree0x100a04120branch0 -> tree0x100b040d0branch0 [style=dashed, arrowhead=lnormal];
	tree0x100a04120branch1 [label="1", shape=circle, style=filled, fillcolor="Grey95"];
	tree0x100a04120branch1 -> tree0x100b04160branch0 [style=dashed, arrowhead=rnormal];
	tree0x100a04120branch1 -> tree0x100a04120branch2 [arrowhead=lnormal];
	tree0x100a04120branch2 [label="1", shape=circle, style=filled, fillcolor="Grey95"];
	tree0x100a04120branch2 -> tree0x100a04120leaf1 [color=Gray75, arrowhead=rnormal];
	tree0x100a04120branch2 -> tree0x100a04120leaf2 [color=Gray75, arrowhead=lnormal];
	tree0x100a04120leaf1 [label = <Fomalhaut<font color="Gray75">\detokenize{⊔}</font>>];
	tree0x100a04120leaf2 [label = <Gacrux<font color="Gray75">\detokenize{⊔}</font>>];
	tree0x100b04160branch0 [label="0", shape=circle, style=filled, fillcolor="Grey95"];
	tree0x100b04160branch0 -> tree0x100b04160leaf0 [color=Gray75, arrowhead=rnormal];
	tree0x100b04160branch0 -> tree0x100b04160branch1 [arrowhead=lnormal];
	tree0x100b04160branch1 [label="0", shape=circle, style=filled, fillcolor="Grey95"];
	tree0x100b04160branch1 -> tree0x100b04160leaf1 [color=Gray75, arrowhead=rnormal];
	tree0x100b04160branch1 -> tree0x100b04160branch2 [arrowhead=lnormal];
	tree0x100b04160branch2 [label="11", shape=circle, style=filled, fillcolor="Grey95"];
	tree0x100b04160branch2 -> tree0x100b04160leaf2 [color=Gray75, arrowhead=rnormal];
	tree0x100b04160branch2 -> tree0x100b04160branch3 [arrowhead=lnormal];
	tree0x100b04160branch3 [label="2", shape=circle, style=filled, fillcolor="Grey95"];
	tree0x100b04160branch3 -> tree0x100b04160leaf3 [color=Gray75, arrowhead=rnormal];
	tree0x100b04160branch3 -> tree0x100b04160leaf4 [color=Gray75, arrowhead=lnormal];
	tree0x100b04160leaf0 [label = <Altair<font color="Gray75">\detokenize{⊔}</font>>];
	tree0x100b04160leaf1 [label = <Betelgeuse<font color="Gray75">\detokenize{⊔}</font>>];
	tree0x100b04160leaf2 [label = <Canopus<font color="Gray75">\detokenize{⊔}</font>>];
	tree0x100b04160leaf3 [label = <Capella<font color="Gray75">\detokenize{⊔}</font>>];
	tree0x100b04160leaf4 [label = <Castor<font color="Gray75">\detokenize{⊔}</font>>];
	tree0x100b040d0branch0 [label="1", shape=circle, style=filled, fillcolor=Grey95];
	tree0x100b040d0branch0 -> tree0x100b040d0branch1 [arrowhead=rnormal];
	tree0x100b040d0branch0 -> tree0x100b040d0leaf6 [color=Gray75, arrowhead=lnormal];
	tree0x100b040d0branch1 [label="0", shape=circle, style=filled, fillcolor=Grey95];
	tree0x100b040d0branch1 -> tree0x100b040d0branch2 [arrowhead=rnormal];
	tree0x100b040d0branch1 -> tree0x100b040d0branch3 [arrowhead=lnormal];
	tree0x100b040d0branch2 [label="21", shape=circle, style=filled, fillcolor=Grey95];
	tree0x100b040d0branch2 -> tree0x100b040d0leaf0 [color=Gray75, arrowhead=rnormal];
	tree0x100b040d0branch2 -> tree0x100b040d0leaf1 [color=Gray75, arrowhead=lnormal];
	tree0x100b040d0branch3 [label="0", shape=circle, style=filled, fillcolor=Grey95];
	tree0x100b040d0branch3 -> tree0x100b040d0branch4 [arrowhead=rnormal];
	tree0x100b040d0branch3 -> tree0x100b040d0branch5 [arrowhead=lnormal];
	tree0x100b040d0branch4 [label="4", shape=circle, style=filled, fillcolor=Grey95];
	tree0x100b040d0branch4 -> tree0x100b040d0leaf2 [color=Gray75, arrowhead=rnormal];
	tree0x100b040d0branch4 -> tree0x100b040d0leaf3 [color=Gray75, arrowhead=lnormal];
	tree0x100b040d0branch5 [label="3", shape=circle, style=filled, fillcolor=Grey95];
	tree0x100b040d0branch5 -> tree0x100b040d0leaf4 [color=Gray75, arrowhead=rnormal];
	tree0x100b040d0branch5 -> tree0x100b040d0leaf5 [color=Gray75, arrowhead=lnormal];
	tree0x100b040d0leaf0 [label = <Polaris<font color="Gray75">\detokenize{⊔}</font>>];
	tree0x100b040d0leaf1 [label = <Pollux<font color="Gray75">\detokenize{⊔}</font>>];
	tree0x100b040d0leaf2 [label = <Regulus<font color="Gray75">\detokenize{⊔}</font>>];
	tree0x100b040d0leaf3 [label = <Rigel<font color="Gray75">\detokenize{⊔}</font>>];
	tree0x100b040d0leaf4 [label = <Sirius<font color="Gray75">\detokenize{⊔}</font>>];
	tree0x100b040d0leaf5 [label = <Spica<font color="Gray75">\detokenize{⊔}</font>>];
	tree0x100b040d0leaf6 [label = <Vega<font color="Gray75">\detokenize{⊔}</font>>];
}
	}
	\caption{A trie as a forest of 3 trees of order-7.\label{star-1}}
\end{figure}

The grey \code{Altair} and \code{Polaris} in the root tree, \code{Vakgimbat}, in Figure~\ref{star-1}, are samples of the the trees that are links. We could get any sample from the sub-tree, because all the bits up to bit 6 and 4, respectively, are the same in the sub-tree. Any time we are Faced with an ambiguity, we arbitrarily and conveniently select the very left.

Picking a sample is also important when calculating asymptotic run time of adding keys. The worse case would be an engineered trie, (not with randomly distributed keys,) which with many left-links at once. On addition of a short key to the right, the algorithm must go though all the left-links to arrive at a key for comparison. This leads to worst-case performance $\mathcal{O}(|\text{trie}|)$, something we don't see in practice. We argue that the length of a key should be amortized for future samples in insertions; while there can be multiple insertions with the same trie structure, it becomes less important as the trie grows. Amortized $\mathcal{O}(|\text{key}|)$ is more what we see in practice, which is related to amortized $\mathcal{O}(\log |\text{trie}|)$\cite{shannon1948mathematical}.

\subsection{Inserting and deleting keys}

To add a key to an existing trie, first we match the key's bits with the tree. If it doesn't have enough length to pick out one key, we arbitrarily choose the left-most alphabetically. We call this the ...

It was tried doing this is one pass by updating the sample . . .

\subsection{Hysteresis}

The non-empty criteria of the trees avoids the pathological case where empty trees from deletion pop-up. Further, we can always join a single leaf with its parent except the in the root.

With smaller, dynamic tries, it is more important to not free resources which could be used in the future. Anything less than greedy merging on deletion will have hysteresis. We also should have a zero-key-state with resources.

\subsection{Data size and order}

We will push the index to be as small as possible, but no more. The order, or branching factor, is the number of leaves, which is bounded by $\max(\code{left}) + 2$. We should have a zero-length flag on the length for empty but active. This is not onerous because the alignment supports a size, then $2^n-1$ index entries, then $2^n$ leaves and bits in the bitmap.

\section{Analysis}

\subsection{Run-time}

\cite{tong2016smoothed}

\subsection{Size of a tree}

\begin{figure}%
\centering%
\begin{gnuplot}[terminal=cairolatex, terminaloptions={color dashed pdf size 6.2,3.4}]
set key font "modern,20"
set grid
set monochrome
set xlabel "order"
set ylabel 'time {\it per} key (µs)'
set format x '\tiny %g'
set format y '\tiny %g'
set key font "modern,20"
set xrange [0:256]
plot \
"timing4.tsv" using 1:($3/$2):($4/$2) with errorlines title "10000 keys", \
"timing5.tsv" using 1:($3/$2):($4/$2) with errorlines title "100000 keys", \
"timing6.tsv" using 1:($3/$2):($4/$2) with errorlines title "1000000 keys"
\end{gnuplot}
\caption{The effects of order on run-time.}%
\label{timing}%
\end{figure}%

Figure~\ref{timing} shows straight insertion of different numbers of keys. It uses two-octet size for each of the branches on the index, divided evenly between \code{left} and \code{skip}. The order is how many leaves each tree holds, either keys or links.

The smaller the order, the more links; this adversely affects the performance because the contents of the next index must be fetched into cache, and the trees split more often. The larger the order, the more updates to the local tree on insertion.\cite{sinha2004cache} In Figure~\ref{timing}, we see a very shallow maximum performance, corresponding to a minimum time. However, at low orders, the performance noticeably suffers. Specifically, when we don't fill 64\,kB of our cache lines.

\subsection{Performance}

\begin{figure}\centering
\begin{subcaptionblock}{\textwidth}
\centering
\begin{gnuplot}[terminal=cairolatex, terminaloptions={color dashed pdf size 6.2,3.4}]
set format x '\tiny %g'
set format y '\tiny %g'
set key font "modern,20"
set grid
set monochrome
set xlabel "keys"
set ylabel 'amortized time {\it per} key, (µs)'
set yrange [0:0.5]
set xrange [1:8388608]
set log x
plot \
"trie-look.tsv" using 1:($2/$1):($3/$1) with errorlines title "trie look", \
"tree-look.tsv" using 1:($2/$1):($3/$1) with errorlines title "tree look", \
"table-look.tsv" using 1:($2/$1):($3/$1) with errorlines title "hash table look"
\end{gnuplot}
\caption{Time to lookup all keys.}
\label{compare:look}
\end{subcaptionblock}
\begin{subcaptionblock}{\textwidth}
\centering
\begin{gnuplot}[terminal=cairolatex, terminaloptions={color dashed pdf size 6.2,3.4}]
set format x '\tiny %g'
set format y '\tiny %g'
set key font "modern,20"
set grid
set monochrome
set xlabel "keys"
set ylabel 'amortized time {\it per} key, (µs)'
set yrange [0:3]
set xrange [1:8388608]
set log x
plot \
"trie-add.tsv" using 1:($2/$1):($3/$1) with errorlines title "trie add", \
"tree-add.tsv" using 1:($2/$1):($3/$1) with errorlines title "tree add", \
"table-add.tsv" using 1:($2/$1):($3/$1) with errorlines title "hash table add"
\end{gnuplot}
\caption{Time to add all keys.}
\label{compare:add}
\end{subcaptionblock}
\caption{Comparison of look-up and insertion in three different data structures.}%
\label{compare}%
\end{figure}%

Figure~\ref{compare}\ldots But compare lookup, too.

The application decides the data-structure. For an unordered set, a hash table is still hard to beat. If order is needed, then a B-tree. This is seen practice in, for example, \code{C$^{++}$23} where \code{unordered\_set} is commonly a hash table and \code{std::ordered\_set} is a red-black tree. On top of that, if prefix matching is at all convenient, a Patricia trie is a really competitive solution.

\section{Conclusion}

It's okay.

\bibliography{trie}

\end{document}
